% !TEX program = xelatex
\PassOptionsToPackage{quiet}{fontspec}
\documentclass[fontset=source]{ctexart}
\usepackage{tcolorbox.doc.s_main}
\usepackage{indentfirst}
\setlength{\parindent}{2\ccwd}
\usepackage{shortvrb}
\usepackage[minted]{macbox}
\usepackage{zhlipsum}
\tcbuselibrary{documentation}
\begin{document}
\begin{macwindow1}[title={MacOS 风格的文本框},fonttitle=\huge\bfseries\color{spot},breakable,colback=page]
    \begin{center}
        \vskip 1cm
        张泓知\par\vskip1ex
        \zhtoday\vskip.5cm
        \begin{macwindow2}[title={摘\quad 要}]
            在“LaTeX技术交流1群”(群号:91940767)的一次交流中做了一个MacOS风格的代码块。现在把其中的一段代码抽出来专门做成一个宏包。
        \end{macwindow2}
    \end{center}
    \par\vspace*{.5cm}
    上面的标题和摘要部分,纯粹是为了展示文本框,并不代表建议大家在标题和摘要部分使用这样的排版。下面是几个示例。
    \begin{maclisting1}[title={最小代码示例}]
        \documentclass[fontset=fandol]{ctexart}
        \usepackage{macbox}
        \usepackage{zhlipsum}
        \begin{document}
        \begin{macwindow2}[title={《祝福》(选段)——鲁迅【著】}]
            \zhlipsum[18][name=zhufu]
        \end{macwindow2}
        \end{document}
    \end{maclisting1}
    那么就会得到下面这样的文本框:
    \begin{macwindow2}[title={《祝福》(选段)——鲁迅【著】}]
        \zhlipsum[18][name=zhufu]
    \end{macwindow2}
    第二个示例:
    \begin{maclisting1}[title={第二个示例},listing side text]
        \begin{macwindow2}[title={静夜思}]
            床前明月光,疑似地上霜。

            举头望明月,低头思故乡。
        \end{macwindow2}
    \end{maclisting1}
\end{macwindow1}


\section{MacOS风格的文本框的绘制代码}
该文本框基于\texttt{tcolorbox}的\mylib{skins}库,结合{\ttfamily tikz}宏包的绘制代码生成的。首先,我们需要在文档的导言区添加下面几行代码。
\begin{maclisting1}[title={导言区的内容}]
    \usepackage{fontawesome}
    \usepackage{tcolorbox}
    \tcbuselibrary{skins}
    \definecolor{icon}{HTML}{aa937c} % 标题前的图标颜色
    \definecolor{wt1}{HTML}{ebebeb} % 白色主题的标题框顶部背景颜色
    \definecolor{wt2}{HTML}{bebebe} % 白色主题的标题框底部背景颜色
    \definecolor{wt3}{HTML}{efefef} % 白色主题的正文部分的背景颜色
    \definecolor{blk1}{HTML}{5e5e60} % 黑色主题的标题框顶部背景颜色
    \definecolor{blk2}{HTML}{2f3032} % 黑色主题的标题框底部背景颜色
    \definecolor{blk3}{HTML}{363739} % 黑色主题的正文部分的背景颜色
    \definecolor{circ1}{HTML}{eb605b} % 标题框左边第一个圆圈的颜色
    \definecolor{circ2}{HTML}{f6bb31} % 标题框左边第二个圆圈的颜色
    \definecolor{circ3}{HTML}{56cb45} % 标题框左边第三个圆圈的颜色
    % 如果需要像后面例子中一样用到\zhlipsum的命令就再加上下面这一句
    \usepackage{zhlipsum}
\end{maclisting1}
其中,最后一行{\verb|\usepackage{zhlipsum}|}并不是必须的,只是出于演示文档生成随机文字的需要才添加的。

文本框的绘图部分主要是利用了\mylib{skins}库中的{\ttfamily enhanced}皮肤,通过对其中的\docAuxKey[tcb]{frame code}进行重画,得到了下面的黑白两种风格的
文本框。
\begin{maclisting1}[title={白色框的代码},listing and text,colback=page]
    \begin{tcolorbox}[
            enhanced,
            boxrule=0pt,
            frame code={
                    \clip[rounded corners=1mm] (frame.south west) rectangle (frame.north east);
                    \shade[top color=wt1,bottom color=wt2] (title.south west) rectangle (title.north east);
                    \fill[circ1] ([xshift=  2em]title.west) circle [radius=1ex];
                    \fill[circ2] ([xshift=3.5em]title.west) circle [radius=1ex];
                    \fill[circ3] ([xshift=  5em]title.west) circle [radius=1ex];
                },
            before title={\textcolor{icon}{\faHome}},
            halign title=flush center,
            title = {白色风格},
            before upper={\setlength{\parindent}{2\ccwd}},
            fonttitle=\sffamily\zihao{-5},
            coltitle = black,
            coltext = black,
            colback = wt3,
        ]
        \zhlipsum[1][name=zhufu]
    \end{tcolorbox}
\end{maclisting1}

\begin{maclisting1}[title={黑色框的代码},listing and text]
    \begin{tcolorbox}[
            enhanced,
            boxrule=0pt,
            frame code={
                    \clip[rounded corners=1mm] (frame.south west) rectangle (frame.north east);
                    \shade[top color=blk1,bottom color=blk2] (title.south west) rectangle (title.north east);
                    \fill[circ1] ([xshift=  2em]title.west) circle [radius=1ex];
                    \fill[circ2] ([xshift=3.5em]title.west) circle [radius=1ex];
                    \fill[circ3] ([xshift=  5em]title.west) circle [radius=1ex];
                },
            before title={\textcolor{icon}{\faHome}},
            halign title=flush center,
            title = {黑色风格},
            before upper={\setlength{\parindent}{2\ccwd}},
            fonttitle=\sffamily\zihao{-5},
            coltitle = white,
            coltext = white,
            colback = blk3,
        ]
        \zhlipsum[2][name=zhufu]
    \end{tcolorbox}
\end{maclisting1}

\section{使用{\ttfamily macbox}宏包}
使用{\ttfamily macbox}宏包可以更轻松地获得相同的效果。只需要把{\ttfamily macbox.sty}文件放在同目录下,然后把前面导言区的内容换成如下即可。
\begin{maclisting1}[title={新的导言区内容}]
    \usepackage{macbox}
    % \usepackage[minted]{macbox}
\end{maclisting1}
{\ttfamily macbox}宏包提供了四种盒子,分别是

\begin{docEnvironments}
    [
        doc no index,
        doc parameter               =   \oarg{options},
        doclang/environment content =   {盒子内容或代码\ },
    ]
    {
        {
                doc name            =   macwindow1,
                doc description     =   {白色风格的窗格},
            },
        {
                doc name            =   macwindow2,
                doc description     =   {黑色风格的窗格},
            },
        {
                doc name            =   maclisting1,
                doc description     =   {白色风格的代码框},
            },
        {
                doc name            =   maclisting2,
                doc description     =   {黑色风格的代码框},
            }
    }
\end{docEnvironments}

下面是几个示例,把窗口的代码分别放在下方对应的代码框里:
\begin{tcbraster}[raster columns=2,raster equal height=rows]
    \begin{macwindow1}
        \zhlipsum[20][name=zhufu]
    \end{macwindow1}
    \begin{macwindow2}
        \zhlipsum[20][name=zhufu]
    \end{macwindow2}
    \begin{maclisting1}
        \begin{macwindow1}
            \zhlipsum[20][name=zhufu]
        \end{macwindow1}
    \end{maclisting1}
    \begin{maclisting2}
        \begin{macwindow2}
            \zhlipsum[20][name=zhufu]
        \end{macwindow2}
    \end{maclisting2}
\end{tcbraster}

另外,本宏包还提供了一个宏包选项 \docAuxKey{minted} :
\begin{docKey*}{minted}{\colOpt{=true\textbar false}}{默认值{\ttfamily false},初始值{\ttfamily false}}
    \setlength{\parindent}{2\ccwd}
    可以简写为 \docAuxKey{minted} 代表 \docAuxKey{minted}\colOpt{\ttfamily =true} ,或直接省略
    代表 \docAuxKey{minted}\colOpt{\ttfamily =false} 。顾名思义,这是为了其中两个代码框环境加载{\ttfamily minted}宏包。

    {\ttfamily minted}宏包是在{\ttfamily listings}宏包的基础上,利用{\ttfamily Python}的{\ttfamily Pygments}包的强大能力,
    把代码高亮及美化更进一步,有更加灵活和多样的变化,就像在一个真正的代码编辑器中浏览代码一样。
\end{docKey*}

要使用{\ttfamily minted}宏包,需要先在系统上安装 {\ttfamily Python} 环境以及 {\ttfamily pip} 工具( {\ttfamily Python 3.4+}
以上版本都自带 {\ttfamily pip} 工具),然后打开命令行运行下列命令:
\begin{maclisting2}[minted language=shell,title={命令行窗口}]
    pip install pygments
\end{maclisting2}
一旦安装完成,即可打开 \docAuxKey{minted} 宏包选项。也就是在前面“新的导言区内容”中,把第一行的代码{\ \verb|\usepackage{macbox}|\ }
注释掉,把第二行代码{\ \verb|% \usepackage[minted]{macbox}|\ }取消注释。然后编译源文件的时候,带上 {\ttfamily -shell-escape} 选项,
在命令行窗口输入下列命令即可。
\begin{maclisting2}[minted language=shell,title={命令行窗口}]
    xelatex -shell-escape macbox.tex
\end{maclisting2}

虽然使用{\ttfamily minted}宏包略有麻烦,但是带来的效果也是极棒的。下面比较两段代码(一段是C++代码,一段是Python代码)分别在不同引擎格式
下的美化效果,
\begin{maclisting1}[minted language=cpp,title={Hello World!(minted 美化的C++代码)}]
    #include <iostream>
    using namespace std;

    // main() 是程序开始执行的地方

    int main()
    {
            cout << "Hello World!"; // 输出 Hello World!
            return 0;
        }
\end{maclisting1}

\begin{maclisting1}[listing engine=listings,listing options={style=tcblatex,language=C++,},title={Hello World!(listings 美化的C++代码)}]
    #include <iostream>
    using namespace std;

    // main() 是程序开始执行的地方

    int main()
    {
            cout << "Hello World!"; // 输出 Hello World!
            return 0;
        }
\end{maclisting1}
再看看Python代码在不同引擎下的美化效果:

\begin{tcbraster}[raster columns=2,raster equal height=rows]
    \begin{maclisting1}[minted language=python,title={Hello, Python!(minted)}]
        #!/usr/bin/python

        print ("Hello, Python!")
    \end{maclisting1}
    \begin{maclisting1}[listing engine=listings,listing options={style=tcblatex,language=Python,},title={Hello, Python!(listings)}]
        #!/usr/bin/python

        print ("Hello, Python!")
    \end{maclisting1}
    % \begin{maclisting2}[minted language=python,title={Hello, Python!(minted)}]
    %     #!/usr/bin/python

    %     print ("Hello, Python!")
    % \end{maclisting2}
    % \begin{maclisting2}[listing engine=listings,listing options={style=tcblatex,language=Python,},title={Hello, Python!(listings)}]
    %     #!/usr/bin/python

    %     print ("Hello, Python!")
    % \end{maclisting2}
\end{tcbraster}

\section{其它示例}
这里给一个用 \docAuxEnvironment*{macwindow1} 来模拟一个 MacOS 的文件夹窗口的例子。
\begin{maclisting1}[title={模拟 MacOS 文件夹窗口},colback=page]
    \begin{macwindow1}[title={文件夹},before title={\textcolor[HTML]{373739}{\faWhmcs\ }}]
        \folder{\faDropbox}{Archiver}
        \folder{\faChrome}{Camera}
        \folder{\faVideo}{Record Screen}
        \folder{\faCamera}{Screenshots}
        \folder{\faClock}{Time}
    \end{macwindow1}
\end{maclisting1}
\begin{macwindow1}[title={文件夹},before title={\textcolor[HTML]{373739}{\faWhmcs\ }}]
    \folder{\faDropbox}{Archiver}
    \folder{\faChrome}{Camera}
    \folder{\faVideo}{Record Screen}
    \folder{\faCamera}{Screenshots}
    \folder{\faClock}{Time}
\end{macwindow1}
\end{document}
